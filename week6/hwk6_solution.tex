%----------------------------------------------------------------------------------------
%	PACKAGES AND OTHER DOCUMENT CONFIGURATIONS
%----------------------------------------------------------------------------------------

\documentclass[paper=a4, fontsize=11pt]{scrartcl} % A4 paper and 11pt font size

\usepackage[T1]{fontenc} % Use 8-bit encoding that has 256 glyphs
\usepackage{fourier} % Use the Adobe Utopia font for the document - comment this line to return to the LaTeX default
%\usepackage{palatino}
\usepackage[english]{babel} % English language/hyphenation
\usepackage{amsmath,amsfonts,amsthm} % Math packages

\usepackage{enumerate}
\usepackage{listings}  % Use this for better code formatting
%\usepackage{xcolor} 
\usepackage{graphicx}    
\usepackage[svgnames]{xcolor} 

%\usepackage{sectsty} % Allows customizing section commands
%\allsectionsfont{\centering \normalfont\scshape} % Make all sections centered, the default font and small caps
 \usepackage{setspace} 
 
\usepackage{fancyhdr} % Custom headers and footers
\pagestyle{fancyplain} % Makes all pages in the document conform to the custom headers and footers
\fancyhead{} % No page header - if you want one, create it in the same way as the footers below
\fancyfoot[L]{} % Empty left footer
\fancyfoot[C]{} % Empty center footer
\fancyfoot[R]{\thepage} % Page numbering for right footer
\renewcommand{\headrulewidth}{0pt} % Remove header underlines
\renewcommand{\footrulewidth}{0pt} % Remove footer underlines
\setlength{\headheight}{13.6pt} % Customize the height of the header
\setlength{\parindent}{0cm}
\setlength{\parskip}{.1cm plus4mm minus3mm}


%--- color definitions for code listings
\definecolor{mygreen}{rgb}{0,0.6,0}
\definecolor{mygray}{rgb}{0.5,0.5,0.5}
\definecolor{mymauve}{rgb}{0.58,0,0.82}


% commands I use to typeset matrices in bold
\newcommand{\matLambda}{\mathbf{\Lambda}}
\newcommand{\matSigma}{\mathbf{\Sigma}}
\newcommand{\vecBeta}{\mathbf{\beta}}
\newcommand{\vecEpsilon}{\mathbf{\epsilon}}
\newcommand{\vecMu}{\mathbf{\mu}}
\newcommand{\vecX}{\mathbf{x}}
\newcommand{\vecY}{\mathbf{y}}
\newcommand{\matA}{\mathbf{A}}
\newcommand{\matB}{\mathbf{B}}
\newcommand{\matC}{\mathbf{C}}
\newcommand{\matI}{\mathbf{I}}
\newcommand{\matM}{\mathbf{M}}
\newcommand{\matP}{\mathbf{P}}
\newcommand{\matS}{\mathbf{S}}
\newcommand{\matV}{\mathbf{V}}
\newcommand{\matQ}{\mathbf{Q}}
\newcommand{\matX}{\mathbf{X}}
\newcommand{\matY}{\mathbf{Y}}
\newcommand{\matW}{\mathbf{W}}

\setcounter{MaxMatrixCols}{20}

%\numberwithin{equation}{section} % Number equations within sections (i.e. 1.1, 1.2, 2.1, 2.2 instead of 1, 2, 3, 4)
%\numberwithin{figure}{section} % Number figures within sections (i.e. 1.1, 1.2, 2.1, 2.2 instead of 1, 2, 3, 4)
%\numberwithin{table}{section} % Number tables within sections (i.e. 1.1, 1.2, 2.1, 2.2 instead of 1, 2, 3, 4)

%\setlength\parindent{0pt} % Removes all indentation from paragraphs - comment this line for an assignment with lots of text

%----------------------------------------------------------------------------------------
%	TITLE SECTION
%----------------------------------------------------------------------------------------

\newcommand{\horrule}[1]{\rule{\linewidth}{#1}} % Create horizontal rule command with 1 argument of height

\title{	
\normalfont \normalsize 
%\textsc{stat 8004} \\ [25pt] 
\horrule{0.5pt} \\[0.4cm] % Thin top horizontal rule
\huge STAT 8004, Assignment 4 \\ % The assignment title
\horrule{2pt} \\[0.5cm] % Thick bottom horizontal rule
}

\author{David Dobor} 

\date{\normalsize\today} % Today's date or a custom date

\lstset{ %
  language=R,                     % the language of the code
  basicstyle=\small\ttfamily,
  backgroundcolor=\color{WhiteSmoke},  % choose the background color. You must add \usepackage{color}
  showspaces=false,               % show spaces adding particular underscores
  showstringspaces=false,         % underline spaces within strings
  showtabs=false,                 % show tabs within strings adding particular underscores
  frame=single,                   % adds a frame around the code
  rulecolor=\color{Gray},        % if not set, the frame-color may be changed on line-breaks within not-black text (e.g. commens (green here))
  tabsize=2,                      % sets default tabsize to 2 spaces
  captionpos=b,                   % sets the caption-position to bottom
  breaklines=true,                % sets automatic line breaking
  breakatwhitespace=false,        % sets if automatic breaks should only happen at whitespace
  title=\lstname,                 % show the filename of files included with \lstinputlisting;
                                  % also try caption instead of title
  keywordstyle=\color{DarkSlateBlue},      % keyword style
  commentstyle=\color{ForestGreen},   % comment style
  stringstyle=\color{mymauve},      % string literal style
  %escapeinside={\%*}{*)},         % if you want to add a comment within your code
  morekeywords={*,det,...}            % if you want to add more keywords to the set
} 

\renewcommand{\ttdefault}{pcr}  %to be able to use bold fonts in lstlistings code

\begin{document}

\maketitle 

%----------------------------------------------------------------------------------------
%	PROBLEM 1
%----------------------------------------------------------------------------------------
\subsection*{Answer to Question 1}
\begin{itemize}
\item Let $\alpha_i$ be the fixed effect of the $i^{\text{th}}$ temperature level,  $i = 1, 2, 3$. 
\item Let $u_{ij}$ be the random effect of the $j^{\text{th}}$ cooler at the  $i^{\text{th}}$ temperature level, $j = 1,2,3,4$.
\item Let $\epsilon_{ijk}$ be the error associated with the $k^{\text{th}}$ cut for the $j^{\text{th}}$ cooler at the  $i^{\text{th}}$ temperature level,  $k = 1, 2$. 
\item Let $y_{ijk}$ be the scores assigned by the putative meat scoring experts, with $i, j, k$ in the range just mentioned.
\end{itemize}

The model is then:

$$
\begin{bmatrix} 
y_{1 1 1} \\ y_{1 1 2}\\ 
y_{1 2 1} \\ y_{1 2 2} \\ 
y_{1 3 1} \\ y_{1 3 2} \\ 
y_{1 4 1} \\ y_{1 4 2} \\ 
y_{2 1 1} \\y_{2 1 2} \\ 
y_{2 2 1} \\ y_{2 2 2} \\
y_{2 3 1} \\ y_{2 3 2} \\ 
y_{2 4 1} \\ y_{2 4 2} \\ 
y_{3 1 1} \\y_{3 1 2} \\ 
y_{3 2 1} \\ y_{3 2 2} \\
y_{3 3 1} \\ y_{3 3 2} \\ 
y_{3 4 1} \\ y_{3 4 2} 
\end{bmatrix}
=
\begin{bmatrix} 1 & 1 & 0 & 0 \\
                             1 & 1 & 0 & 0 \\
                             1 & 1 & 0 & 0 \\
                             1 & 1 & 0 & 0 \\
                             1 & 1 & 0 & 0 \\
                             1 & 1 & 0 & 0 \\
                             1 & 1 & 0 & 0 \\
                             1 & 1 & 0 & 0 \\
                             1 & 0 & 1 & 0 \\
                             1 & 0 & 1 & 0 \\
                             1 & 0 & 1 & 0 \\
                             1 & 0 & 1 & 0 \\
                             1 & 0 & 1 & 0 \\
                             1 & 0 & 1 & 0 \\
                             1 & 0 & 1 & 0 \\
                             1 & 0 & 1 & 0 \\
                             1 & 0 & 0 & 1 \\
                             1 & 0 & 0 & 1 \\
                             1 & 0 & 0 & 1 \\
                             1 & 0 & 0 & 1 \\
                             1 & 0 & 0 & 1 \\
                             1 & 0 & 0 & 1 \\
                             1 & 0 & 0 & 1 \\
                             1 & 0 & 0 & 1 
\end{bmatrix}
\begin{bmatrix} 
\mu \\ \alpha_1\\ \alpha_2 \\ \alpha_3
\end{bmatrix}
+
\begin{bmatrix} 1 & 0 & 0 & 0 & 0 & 0 & 0 & 0 & 0 & 0 & 0 & 0\\
                             1 & 0 & 0 & 0 & 0 & 0 & 0 & 0 & 0 & 0 & 0 & 0\\
                             0 & 1 & 0 & 0 & 0 & 0 & 0 & 0 & 0 & 0 & 0 & 0\\
                             0 & 1 & 0 & 0 & 0 & 0 & 0 & 0 & 0 & 0 & 0 & 0\\
                             0 & 0 & 1 & 0 & 0 & 0 & 0 & 0 & 0 & 0 & 0 & 0\\
                             0 & 0 & 1 & 0 & 0 & 0 & 0 & 0 & 0 & 0 & 0 & 0\\
                             0 & 0 & 0 & 1 & 0 & 0 & 0 & 0 & 0 & 0 & 0 & 0\\
                             0 & 0 & 0 & 1 & 0 & 0 & 0 & 0 & 0 & 0 & 0 & 0\\
                             0 & 0 & 0 & 0 & 1 & 0 & 0 & 0 & 0 & 0 & 0 & 0\\
                             0 & 0 & 0 & 0 & 1 & 0 & 0 & 0 & 0 & 0 & 0 & 0\\
                             0 & 0 & 0 & 0 & 0 & 1 & 0 & 0 & 0 & 0 & 0 & 0\\
                             0 & 0 & 0 & 0 & 0 & 1 & 0 & 0 & 0 & 0 & 0 & 0\\
                             0 & 0 & 0 & 0 & 0 & 0 & 1 & 0 & 0 & 0 & 0 & 0\\
                             0 & 0 & 0 & 0 & 0 & 0 & 1 & 0 & 0 & 0 & 0 & 0\\
                             0 & 0 & 0 & 0 & 0 & 0 & 0 & 1 & 0 & 0 & 0 & 0\\
                             0 & 0 & 0 & 0 & 0 & 0 & 0 & 1 & 0 & 0 & 0 & 0\\
                             0 & 0 & 0 & 0 & 0 & 0 & 0 & 0 & 1 & 0 & 0 & 0\\
                             0 & 0 & 0 & 0 & 0 & 0 & 0 & 0 & 1 & 0 & 0 & 0\\
                             0 & 0 & 0 & 0 & 0 & 0 & 0 & 0 & 0 & 1 & 0 & 0\\
                             0 & 0 & 0 & 0 & 0 & 0 & 0 & 0 & 0 & 1 & 0 & 0\\
                             0 & 0 & 0 & 0 & 0 & 0 & 0 & 0 & 0 & 0 & 1 & 0\\
                             0 & 0 & 0 & 0 & 0 & 0 & 0 & 0 & 0 & 0 & 1 & 0\\
                             0 & 0 & 0 & 0 & 0 & 0 & 0 & 0 & 0 & 0 & 0 & 1\\
                             0 & 0 & 0 & 0 & 0 & 0 & 0 & 0 & 0 & 0 & 0 & 1
\end{bmatrix}
\begin{bmatrix} 
u_{11}  \\ u_{12} \\ u_{13}  \\ u_{14}  \\ u_{21}  \\ u_{22}  \\ u_{23} \\ u_{24} \\ u_{31}  \\ u_{32}  \\ u_{33} \\ u_{34} 
\end{bmatrix}
+ 
\begin{bmatrix} 
\epsilon_{1 1 1} \\ \epsilon_{1 1 2}\\ 
\epsilon_{1 2 1} \\ \epsilon_{1 2 2} \\ 
\epsilon_{1 3 1} \\ \epsilon_{1 3 2} \\ 
\epsilon_{1 4 1} \\ \epsilon_{1 4 2} \\ 
\epsilon_{2 1 1} \\\epsilon_{2 1 2} \\ 
\epsilon_{2 2 1} \\ \epsilon_{2 2 2} \\
\epsilon_{2 3 1} \\ \epsilon_{2 3 2} \\ 
\epsilon_{2 4 1} \\ \epsilon_{2 4 2} \\ 
\epsilon_{3 1 1} \\\epsilon_{3 1 2} \\ 
\epsilon_{3 2 1} \\ \epsilon_{3 2 2} \\
\epsilon_{3 3 1} \\ \epsilon_{3 3 2} \\ 
\epsilon_{3 4 1} \\ \epsilon_{3 4 2} 
\end{bmatrix}
.
$$\\

Or, to be a bit more space-saving about the whole thing, 
$$
y_{ijk} = \mu + \alpha_i + u_{ij} +\epsilon_{ijk} \ \ \ \text{for } 1 \leq i \leq 3, 1 \leq j \leq 4, 1 \leq k \leq 2.
$$ 


\bigskip
\bigskip
%----------------------------------------------------------------------------------------
%	PROBLEM 2
%----------------------------------------------------------------------------------------
\subsection*{Question 2}
The model $\vecY = \matX \vecMu + \vecEpsilon$ here is 
$$
\begin{bmatrix} 
y_{1 } \\
y_{2} \\
y_{3} 
\end{bmatrix}
= 
\begin{bmatrix} 1 & 0\\
			     1 & 0\\
		 	     0 & 1\\
\end{bmatrix}
\begin{bmatrix} 
\mu_{1 } \\
\mu_{2} 
\end{bmatrix}
+
\begin{bmatrix} 
\epsilon_{1 } \\
\epsilon_{2} \\
\epsilon_{3} 
\end{bmatrix}
$$
With the covariance matrix
$$
\matSigma 
= \sigma^2 
\begin{bmatrix} 1 & 1/2 & 0\\
			     1/2 & 1 & 1/2\\
		 	     0 & 1/2 & 1\\
\end{bmatrix}
$$
We see that 
$$
\text{rank}(X) = 2, \ \ \ \text{ so that } \ \ \ n - \text{rank}(X) = 3 - 2 = 1
$$\\
Consider then the following matrix $\matB$ that satisfies $\matB \; \matX = 0$, with $\text{rank}(\matB) = 1$:
$$
\matB = (1, -1, 0)
$$



Then, according to the discussion at the end of lecture 6\footnote{
For the choice of $\matM$ also discussed in the lecture, we could compute the projection matrices:
$$
\matP_X = 
\begin{bmatrix} 1/2 & 1/2 & 0\\
			     1/2 & 1/2 & 0\\
		 	     0 & 0 & 1\\
\end{bmatrix}
$$
$$
\matI - \matP_X = 
\begin{bmatrix} 1/2 & -1/2 & 0\\
			     -1/2 & 1/2 & 0\\
		 	     0 & 0 & 0\\
\end{bmatrix}
$$
$$
\matM = (1, -1, 0)
$$
Then
$$
\matM (\matI - \matP_X) \vecY =
 (1, -1, 0)
 \begin{bmatrix} 1/2 & -1/2 & 0\\
			     -1/2 & 1/2 & 0\\
		 	     0 & 0 & 0\\
\end{bmatrix}
\vecY
= 
\matB \vecY =
 (1, -1, 0) 
 \begin{bmatrix} y_1\\
			     y_2\\
		 	     y_3 
\end{bmatrix}
 = y_1 - y_2
$$

}, 
$\matB \, \vecY =  (1, -1, 0)
\begin{bmatrix} y_1\\
			     y_2\\
		 	     y_3 
\end{bmatrix} =
y_1 - y_2 $ will be normally distributed with:
\begin{align*}
\matB \, \vecY   &\sim \mathcal{N}(\ 0, \ \matB \, \matSigma \, \matB^T \ )\\
 &\sim \mathcal{N}\Big(\ 0,  \ \sigma^2 
 (1, -1, 0)
\begin{bmatrix} 1 & 1/2 & 0\\
			     1/2 & 1 & 1/2\\
		 	     0 & 1/2 & 1\\
\end{bmatrix}
\begin{bmatrix} 1\\
			     -1\\
		 	     0 
\end{bmatrix} \ \Big)\\
 &\sim \mathcal{N} (\ 0, \ \sigma^2 \ )
\end{align*}

Thus
\begin{center}
\boxed{
$$
y_1 - y_2 \sim \mathcal{N}(\ 0, \ \sigma^2 \ )
$$
}
\end{center}
\bigskip
\bigskip
Now compute the log-likelihood and maximize it by setting its derivative with respect to $\sigma^2$ to zero:

\begin{align*}
l(\sigma^2) &= -\frac{1}{2} \log(2 \pi) -\frac{1}{2} \log(\sigma^2) -\frac{1}{2} \frac{1}{\sigma^2} (y_1 - y_2)^2\\
\frac{d (l(\sigma^2) )} {d \sigma^2} &=  -\frac{1}{2 \sigma^2} + \frac{1}{2 \sigma^4} (y_1 - y_2)^2 = 0\\
\end{align*}
Therefore
\begin{center}
\boxed{
$$
\hat \sigma^2 = (y_1 - y_2)^2 
$$
}
\end{center}

\end{document}