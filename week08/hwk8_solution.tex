%----------------------------------------------------------------------------------------
%	PACKAGES AND OTHER DOCUMENT CONFIGURATIONS
%----------------------------------------------------------------------------------------

\documentclass[paper=a4, fontsize=11pt]{scrartcl} % A4 paper and 11pt font size

\usepackage[T1]{fontenc} % Use 8-bit encoding that has 256 glyphs
\usepackage{fourier} % Use the Adobe Utopia font for the document - comment this line to return to the LaTeX default
%\usepackage{palatino}
\usepackage[english]{babel} % English language/hyphenation
\usepackage{amsmath,amsfonts,amsthm} % Math packages

\usepackage{enumerate}
\usepackage{listings}  % Use this for better code formatting
%\usepackage{xcolor} 
\usepackage{graphicx}    
\usepackage[svgnames]{xcolor} 

\usepackage{pdfpages} %Alows inclusion of external pdf files
\usepackage{float}  %fix table in position

%\usepackage{sectsty} % Allows customizing section commands
%\allsectionsfont{\centering \normalfont\scshape} % Make all sections centered, the default font and small caps
 \usepackage{setspace} 
 
\usepackage{fancyhdr} % Custom headers and footers
\pagestyle{fancyplain} % Makes all pages in the document conform to the custom headers and footers
\fancyhead{} % No page header - if you want one, create it in the same way as the footers below
\fancyfoot[L]{} % Empty left footer
\fancyfoot[C]{} % Empty center footer
\fancyfoot[R]{\thepage} % Page numbering for right footer
\renewcommand{\headrulewidth}{0pt} % Remove header underlines
\renewcommand{\footrulewidth}{0pt} % Remove footer underlines
\setlength{\headheight}{13.6pt} % Customize the height of the header
\setlength{\parindent}{0cm}
\setlength{\parskip}{.1cm plus4mm minus3mm}


%--- color definitions for code listings
\definecolor{mygreen}{rgb}{0,0.6,0}
\definecolor{mygray}{rgb}{0.5,0.5,0.5}
\definecolor{mymauve}{rgb}{0.58,0,0.82}


% commands I use to typeset matrices in bold
\newcommand{\matLambda}{\mathbf{\Lambda}}
\newcommand{\matSigma}{\mathbf{\Sigma}}
\newcommand{\vecBeta}{\mathbf{\beta}}
\newcommand{\vecEpsilon}{\mathbf{\epsilon}}
\newcommand{\vecMu}{\mathbf{\mu}}
\newcommand{\vecX}{\mathbf{x}}
\newcommand{\vecY}{\mathbf{y}}
\newcommand{\matA}{\mathbf{A}}
\newcommand{\matB}{\mathbf{B}}
\newcommand{\matC}{\mathbf{C}}
\newcommand{\matI}{\mathbf{I}}
\newcommand{\matM}{\mathbf{M}}
\newcommand{\matP}{\mathbf{P}}
\newcommand{\matS}{\mathbf{S}}
\newcommand{\matV}{\mathbf{V}}
\newcommand{\matQ}{\mathbf{Q}}
\newcommand{\matX}{\mathbf{X}}
\newcommand{\matY}{\mathbf{Y}}
\newcommand{\matW}{\mathbf{W}}

\setcounter{MaxMatrixCols}{20}

%\numberwithin{equation}{section} % Number equations within sections (i.e. 1.1, 1.2, 2.1, 2.2 instead of 1, 2, 3, 4)
%\numberwithin{figure}{section} % Number figures within sections (i.e. 1.1, 1.2, 2.1, 2.2 instead of 1, 2, 3, 4)
%\numberwithin{table}{section} % Number tables within sections (i.e. 1.1, 1.2, 2.1, 2.2 instead of 1, 2, 3, 4)

%\setlength\parindent{0pt} % Removes all indentation from paragraphs - comment this line for an assignment with lots of text

%----------------------------------------------------------------------------------------
%	TITLE SECTION
%----------------------------------------------------------------------------------------

\newcommand{\horrule}[1]{\rule{\linewidth}{#1}} % Create horizontal rule command with 1 argument of height

\title{	
\normalfont \normalsize 
%\textsc{stat 8004} \\ [25pt] 
\horrule{0.5pt} \\[0.4cm] % Thin top horizontal rule
\huge STAT 8004, Assignment 8 \\ % The assignment title
\horrule{2pt} \\[0.5cm] % Thick bottom horizontal rule
}

\author{David Dobor} 

\date{\normalsize\today} % Today's date or a custom date

\lstset{ %
  language=R,                     % the language of the code
  basicstyle=\small\ttfamily,
  backgroundcolor=\color{WhiteSmoke},  % choose the background color. You must add \usepackage{color}
  showspaces=false,               % show spaces adding particular underscores
  showstringspaces=false,         % underline spaces within strings
  showtabs=false,                 % show tabs within strings adding particular underscores
  frame=single,                   % adds a frame around the code
  rulecolor=\color{Gray},        % if not set, the frame-color may be changed on line-breaks within not-black text (e.g. commens (green here))
  tabsize=2,                      % sets default tabsize to 2 spaces
  captionpos=b,                   % sets the caption-position to bottom
  breaklines=true,                % sets automatic line breaking
  breakatwhitespace=false,        % sets if automatic breaks should only happen at whitespace
  title=\lstname,                 % show the filename of files included with \lstinputlisting;
                                  % also try caption instead of title
  keywordstyle=\color{DarkSlateBlue},      % keyword style
  commentstyle=\color{ForestGreen},   % comment style
  stringstyle=\color{mymauve},      % string literal style
  %escapeinside={\%*}{*)},         % if you want to add a comment within your code
  morekeywords={*,det,...}            % if you want to add more keywords to the set
} 

\renewcommand{\ttdefault}{pcr}  %to be able to use bold fonts in lstlistings code

\begin{document}

\maketitle 

%----------------------------------------------------------------------------------------
%	PROBLEM 1
%----------------------------------------------------------------------------------------
\subsection*{Answer to Question 1}

The problem data:\\
\begin{itemize}
\item 
\end{itemize}

\bigskip
\bigskip
%----------------------------------------------------------------------------------------
%	PROBLEM 2
%----------------------------------------------------------------------------------------
\subsection*{Answer to Question 2}

In this model, for the  $k^{\text{th}}$ piece of the  $j^{\text{th}}$ roll on the  $i^{\text{th}}$ machine we have
$$
y_{ijk} = \mu + \alpha_i + u_{ij} + \epsilon_{ijk},
$$
where all effects except $\mu$ are random.\\ 

This is a \emph{nested} model where $j^{\text{th}}$ piece of roll is nested in the  $i^{\text{th}}$ machine.\\

We have $a = 3$ machines (indexed by $i$), $b = 5$ rolls (indexed by $j$) from each machine, and $c = 5$ pieces (indexed by $k$) from each roll:\\
\begin{table}[H]
\caption{Data for Question 2, Notation Used} % title of Table
\centering % used for centering table
\begin{tabular}{c c c c c c } % centered columns (4 columns)
\hline\hline %inserts double horizontal lines
Source & SS & df  & ( notation for \texttt{df} ) & Mean Squares\\ [0.5ex] % inserts table
%heading
\hline % inserts single horizontal line
Machines & 1966 & $3 - 1 = 2$ & $ = a - 1$ & 1966/2 = 983 = MSA\\ 
Rolls & 644 & $3\cdot(5-1) = 12$  & $ = a(b - 1)$ & 644/12 = 53.667 = MSB(A)\\
Pieces & 280 & $3\cdot5\cdot(5-1) = 60$ & $ = a b (c - 1)$& 280/60 = 4.667 = MSE\\ [0.5ex]
\hline
Total & 2890 & 74  & $abc - 1$\\ [1ex] % [1ex] adds vertical space
\hline %inserts single line
\end{tabular}
\label{table:nonlin} % is used to refer this table in the text
\end{table}


\begin{enumerate}[(a)]
\item  Based on the lecture, we know the following to be true (page 3, lecture 8):
\begin{align*} 
 \mathbf{E}\  (\text{MSE}) &= \sigma^2_\epsilon\\ 
\mathbf{E}\  (\text{MSB(A)})  &= \sigma^2_\epsilon + 5 \cdot \sigma^2_u \\
\mathbf{E}\  (\text{MSA})  &= \sigma^2_\epsilon + 5 \cdot \sigma^2_u + 5\cdot 5 \cdot \sigma^2_\alpha
\end{align*}

Based on these we obtain the \emph{estimates of the variance components:}

\begin{align*} 
\hat \sigma^2_\epsilon &=  4.667\\ 
\hat \sigma^2_u &= \frac{1}{5} (53.667 - 4.667) = 9.8\\
\hat \sigma^2_\alpha &= \frac{1}{25} (983 - 53.667) = 37.1733
\end{align*}

\item 
\begin{itemize}
\item 95\% Confidence Interval for $\sigma^2_\epsilon:$\\

We make use of the fact that $\frac{60 \cdot \texttt{MSE}}{\sigma^2_\epsilon}$ is $\chi^2$ distributed with $60$ degrees of freedom, and compute the 95\% confidence limits for $\sigma^2_\epsilon$:\\
\begin{verbatim}
SSE <- 280
alpha <- .05
lower.limit <- SSE / qchisq(1 - alpha/2, df=60)
upper.limit <- SSE / qchisq(alpha/2, df=60)
CI_error <- sqrt(c(lower.limit, upper.limit))
\end{verbatim}
\bigskip

Thus:
$$
 \boxed{\text{95\% Confidence Interval for } \sigma_\epsilon = ( 1.833423,  2.629961 )}
$$

\bigskip
\item 95\% Confidence Interval for $\sigma^2_u:$\\

We first  use the Cochran -- Saterthwaite procedure to compute the needed degrees of freedom.  Using the results in part a, we have:
$$
\hat \nu = \frac{ (\hat \sigma^2_u)^2 }{ \frac{(\text{MSB(A)} / 5)^2}{12} + \frac{(-\text{MSE}/5)^2}{60} } = 9.988675
$$
The computation of confidence limits is similar to the one shown above, where this $
\hat \nu$ is used for the degrees of freedom for the relevant $\chi^2$ distribution. \\

The full \texttt{R} script that was used to obrain the results is attached in the appendix.\\

Thus:
$$
 \boxed{\text{95\% Confidence Interval for } \sigma_u = ( 2.186967, 5.496051)}
$$
\bigskip
\item 95\% Confidence Interval for $\sigma^2_\alpha:$\\
We do a computation similar to the above. For the degrees of freedom we use:
$$
\hat \nu = \frac{ (\hat \sigma^2_\alpha)^2 }{  \frac{(\text{MSA}/25)^2}{2} + \frac{(-\text{MSB(A)} / 25)^2}{12} } = 1.786695
$$

Thus:
$$
 \boxed{\text{95\% Confidence Interval for } \sigma_\alpha = (3.097864, 46.297890)}
$$
\end{itemize}
\bigskip


\item The 95\% Confidence Interval for $\mu$ is
$$
\left( \ \bar y_{...} - \text{\texttt{qt(0.975,df=2)}} \sqrt{MSA / 75}, \ \  \bar y_{...} + \text{\texttt{qt(0.975,df=2)}} \sqrt{MSA / 75} \ \right)
$$
Which, given $\bar y_{...} = 1/75 \  \sum_{ijk} y_{ijk} = 35$ and MSA from the table, is:
$$
 \boxed{\text{95\% Confidence Interval for } \mu = (19.42305, 50.57695)}
$$
\end{enumerate}
\pagebreak
\section{Appendix}
\begin{verbatim}

################################################################
########################## Question 2 ##########################
################################################################
# from the table:
MSE <- 280/60
MSB_A <- 644/12
MSA <- 1966/2

############################ part(a)############################
sigma2_c <- MSE                   # ans: 4.666667
sigma2_u <- 1/5 * (MSB_A - MSE)   # ans: 9.8
sigma2_a <- 1/25 * (MSA - MSB_A)  # ans: 37.17333

############################ part(b)############################
# for sigma2_c:
alpha <- .05
lower.limit <- 60 * MSE / qchisq(1 - alpha/2, df=60)
upper.limit <- 60 * MSE / qchisq(alpha/2, df=60)
CI_error <- sqrt(c(lower.limit, upper.limit))

# for sigma2_u:
v <- (sigma2_u)^2 / ( ((-MSE/5)^2 / 60)  + (MSB_A/5)^2 / 12 )
lower.limit <- (v*sigma2_u) / qchisq(1 - alpha/2, df=v)
upper.limit <- (v*sigma2_u)/ qchisq(alpha/2, df=v)
CI_u <- sqrt(c(lower.limit, upper.limit))

# for sigma2_a:
v <- (sigma2_a)^2 / ( ((MSA/25)^2 / 2)  + (-MSB_A/25)^2 / 12 )
lower.limit <- (v*sigma2_a) / qchisq(1 - alpha/2, df=v)
upper.limit <- (v*sigma2_a)/ qchisq(alpha/2, df=v)
CI_a <- sqrt(c(lower.limit, upper.limit))

############################ part(c)############################
# 95\% Confidence Interval for \mu:
qt(0.975,df=2)
CI_mu <- c( 35 - qt(0.975,df=2) * sqrt( MSA / 75 ), 35 + qt(0.975,df=2) * sqrt( MSA / 75 ))
CI_mu













\end{verbatim}


%
%The following is a representation of the data for this problem:
%\includepdf[pages={1}]{Quest2DataRepresentation.pdf} 


\end{document}




















