%----------------------------------------------------------------------------------------
%	PACKAGES AND OTHER DOCUMENT CONFIGURATIONS
%----------------------------------------------------------------------------------------

\documentclass[paper=a4, fontsize=11pt]{scrartcl} % A4 paper and 11pt font size

\usepackage[T1]{fontenc} % Use 8-bit encoding that has 256 glyphs
\usepackage{fourier} % Use the Adobe Utopia font for the document - comment this line to return to the LaTeX default
%\usepackage{palatino}
\usepackage[english]{babel} % English language/hyphenation
\usepackage{amsmath,amsfonts,amsthm} % Math packages

\usepackage{enumerate}
\usepackage{listings}  % Use this for better code formatting
%\usepackage{xcolor}    
\usepackage[svgnames]{xcolor} 

%\usepackage{sectsty} % Allows customizing section commands
%\allsectionsfont{\centering \normalfont\scshape} % Make all sections centered, the default font and small caps
 \usepackage{setspace} 
 
\usepackage{fancyhdr} % Custom headers and footers
\pagestyle{fancyplain} % Makes all pages in the document conform to the custom headers and footers
\fancyhead{} % No page header - if you want one, create it in the same way as the footers below
\fancyfoot[L]{} % Empty left footer
\fancyfoot[C]{} % Empty center footer
\fancyfoot[R]{\thepage} % Page numbering for right footer
\renewcommand{\headrulewidth}{0pt} % Remove header underlines
\renewcommand{\footrulewidth}{0pt} % Remove footer underlines
\setlength{\headheight}{13.6pt} % Customize the height of the header
\setlength{\parindent}{0cm}
\setlength{\parskip}{.1cm plus4mm minus3mm}


%--- color definitions for code listings
\definecolor{mygreen}{rgb}{0,0.6,0}
\definecolor{mygray}{rgb}{0.5,0.5,0.5}
\definecolor{mymauve}{rgb}{0.58,0,0.82}


% commands I use to typeset matrices in bold
\newcommand{\matLambda}{\mathbf{\Lambda}}
\newcommand{\matA}{\mathbf{A}}
\newcommand{\matB}{\mathbf{B}}
\newcommand{\matC}{\mathbf{C}}
\newcommand{\matP}{\mathbf{P}}
\newcommand{\matS}{\mathbf{S}}
\newcommand{\matV}{\mathbf{V}}
\newcommand{\matQ}{\mathbf{Q}}
\newcommand{\matX}{\mathbf{X}}
\newcommand{\matY}{\mathbf{Y}}
\newcommand{\matW}{\mathbf{W}}



%\numberwithin{equation}{section} % Number equations within sections (i.e. 1.1, 1.2, 2.1, 2.2 instead of 1, 2, 3, 4)
%\numberwithin{figure}{section} % Number figures within sections (i.e. 1.1, 1.2, 2.1, 2.2 instead of 1, 2, 3, 4)
%\numberwithin{table}{section} % Number tables within sections (i.e. 1.1, 1.2, 2.1, 2.2 instead of 1, 2, 3, 4)

%\setlength\parindent{0pt} % Removes all indentation from paragraphs - comment this line for an assignment with lots of text

%----------------------------------------------------------------------------------------
%	TITLE SECTION
%----------------------------------------------------------------------------------------

\newcommand{\horrule}[1]{\rule{\linewidth}{#1}} % Create horizontal rule command with 1 argument of height

\title{	
\normalfont \normalsize 
%\textsc{stat 8004} \\ [25pt] 
\horrule{0.5pt} \\[0.4cm] % Thin top horizontal rule
\huge STAT 8004, Assignment 2 \\ % The assignment title
\horrule{2pt} \\[0.5cm] % Thick bottom horizontal rule
}

\author{David Dobor} 

\date{\normalsize\today} % Today's date or a custom date

\lstset{ %
  language=R,                     % the language of the code
  basicstyle=\small\ttfamily,
  backgroundcolor=\color{WhiteSmoke},  % choose the background color. You must add \usepackage{color}
  showspaces=false,               % show spaces adding particular underscores
  showstringspaces=false,         % underline spaces within strings
  showtabs=false,                 % show tabs within strings adding particular underscores
  frame=single,                   % adds a frame around the code
  rulecolor=\color{Gray},        % if not set, the frame-color may be changed on line-breaks within not-black text (e.g. commens (green here))
  tabsize=2,                      % sets default tabsize to 2 spaces
  captionpos=b,                   % sets the caption-position to bottom
  breaklines=true,                % sets automatic line breaking
  breakatwhitespace=false,        % sets if automatic breaks should only happen at whitespace
  title=\lstname,                 % show the filename of files included with \lstinputlisting;
                                  % also try caption instead of title
  keywordstyle=\color{DarkSlateBlue},      % keyword style
  commentstyle=\color{ForestGreen},   % comment style
  stringstyle=\color{mymauve},      % string literal style
  %escapeinside={\%*}{*)},         % if you want to add a comment within your code
  morekeywords={*,det,...}            % if you want to add more keywords to the set
} 

\renewcommand{\ttdefault}{pcr}  %to be able to use bold fonts in lstlistings code

\begin{document}

\maketitle 

%----------------------------------------------------------------------------------------
%	PROBLEM 1
%----------------------------------------------------------------------------------------

%\section*{Problem title}

\subsection*{Question 1}
Write out the following models of elementary/intermediate statistical analysis in the matrix form
$$
\matY = \matX \  \beta + \mathbf{\epsilon}
$$
\begin{enumerate}[(a)]
\item A one-variable quadratic polynomial regression model
\begin{align*}
y_i = \alpha_0 + \alpha_1 x_i + \alpha_2 x_i^2 + \epsilon_i
\end{align*}
for $i = 1, 2, \ldots , 5$.
\item A two-factor ANCOVA model without interactions
$$
y_{ijk} = \mu + \alpha_i + \beta_j + \gamma (x_{ijk} - \bar x) + \epsilon_{i j k}
$$
for $i = 1, 2, j = 1, 2, \ \text{and } k = 1, 2.$
\end{enumerate}

\bigskip
\subsubsection*{Answer to Question 1}

%\paragraph{Answer to Question 1}
\begin{enumerate}[(a)]
\item 
$$
\begin{bmatrix} 
y_1 \\ y_2\\ y_3 \\ y_4 \\ y_5 
\end{bmatrix}
=
\begin{bmatrix} 1 & x_{11} & x_{11}^2 \\1 & x_{12} & x_{12}^2 \\1 & x_{13} & x_{13}^2 \\1 & x_{14} & x_{14}^2 \\1 & x_{15} & x_{15}^2 \end{bmatrix}
\begin{bmatrix} 
\alpha_0 \\ \alpha_1\\ \alpha_2
\end{bmatrix}
+ 
\begin{bmatrix} 
\epsilon_1 \\ \epsilon_2\\ \epsilon_3 \\ \epsilon_4 \\ \epsilon_5 
\end{bmatrix}
$$

\item 
$$
\begin{bmatrix} 
y_{1 1 1} \\ y_{1 1 2}\\ y_{1 2 1} \\ y_{1 2 2} \\ y_{2 1 1} \\y_{2 1 2} \\ y_{2 2 1} \\ y_{2 2 2} 
\end{bmatrix}
=
\begin{bmatrix} 1 & 1 & 0 & 1 & 0 & x_{111} - \bar x\\1 & 1 & 0 & 1 & 0 & x_{112} - \bar x\\1 & 1 & 0 & 0 & 1 & x_{121} - \bar x\\1 & 1 & 0 & 0 & 1  & x_{122} - \bar x\\1 & 0 & 1 & 1 & 0 & x_{211} - \bar x\\1 & 0 & 1 & 1 & 0 & x_{212} - \bar x\\1 & 0 & 1 & 0 & 1 & x_{221} - \bar x\\1 & 0 & 1 & 0 & 1 & x_{222} - \bar x\end{bmatrix}
\begin{bmatrix} 
\mu \\ \alpha_1\\ \alpha_2 \\ \beta_1 \\ \beta_2 \\ \gamma 
\end{bmatrix}
+ 
\begin{bmatrix} 
\epsilon_{1 1 1} \\ \epsilon_{1 1 2}\\ \epsilon_{1 2 1} \\ \epsilon_{1 2 2} \\ \epsilon_{2 1 1} \\ \epsilon_{2 1 2} \\ \epsilon_{2 2 1} \\ \epsilon_{2 2 2} 
\end{bmatrix}
$$

\end{enumerate}


\bigskip
\bigskip
%----------------------------------------------------------------------------------------
%	PROBLEM 2
%----------------------------------------------------------------------------------------

\subsection*{Question 2}
Use \texttt{eigen()} function in \texttt{R} to compute the eigenvalues and eigenvectors of
$$
\matV = 
\left( \begin{array}{rrr}
  3 & -1 & 1 \\
  -1 & 5 & -1\\
  1 & -1 & 3
\end{array} \right)
$$
Then use \texttt{R} to find an ``inverse square root'' of this matrix.  That is, find a symmetric matrix $\mathbf{W}$
such that $\matW \matW= \matV^{-1}$

%\paragraph{Answer to Question 2}
\bigskip
\subsubsection*{Answer to Question 2}
$\matV$ is a real symmetric $3 \times 3$ matrix and thus has $3$ real eigenvalues / $3$ orthogonal eigenvectors in $\mathbb{R}^3$; $\matV$ is diagonalizable. We put the eigenvectors into the columns of matrix $\matQ$ in what follows. \footnote{ 
In a more general case when $\matV$ is not symmetric but still has 3 linearly independent eigenvectors, we would still be able to quickly compute the powers of  $\matV$ using the decomposition $\matV = \matS \matLambda \matS^{-1}$, where the columns of $\matS$ contain the eigenvectors of $ \matV$ and $\matLambda$ is the diagonal matrix of eigenvalues. The computation would be similar to what follows, except that we would have to compute the inverse of $\matS$ which is more computationally intensive than simply transposing $\matQ$. In the symmetric case we are fortunate that  $\matQ^T = \matQ^{-1}$ for this orthogonal $\matQ$.}   


Also, $\matV$'s eigenvalues happen to be positive - this is a positive definite matrix.

Thus we can decompose  $\matV$ into the product $\matQ \matLambda \matQ^{T}$, where  $\matLambda$ is the diagonal matrix of eigenvalues and $\matQ$ is the orthogonal matrix containing the eigenvectors in its columns.  We then compute the powers of $\matV$ as follows:

\begin{align*}
\matV^{-1} &= \matQ \matLambda^{-1} \matQ^{T} \\
\matW &= \matQ \matLambda^{-1/2} \matQ^{T} 
\end{align*}

\pagebreak


\begin{lstlisting}[basicstyle=\ttfamily\small\bfseries]
# Solution to question 2

# V, the given matrix:
V <- matrix(c(3, -1, 1, -1, 5, -1, 1, -1, 3), 3) 

# Q (orthogonal here) contains eigenvectors as its columns: 
Q <- eigen(V)$vectors              

# D contains the inverse square roots of of V's eigenvalues 
# on the diagonal, zeros elsewhere:
D <- diag(1/sqrt(eigen(V)$values))

# The desired `inverse square root` matrix:
W <- Q %*% D %*% t(Q)      

\end{lstlisting}


An output produced by \texttt{R}:

\begin{lstlisting}[basicstyle=\ttfamily\small\bfseries]
> options("digits"=4)  #show fewer decimals
> print(W)

         [,1]    [,2]     [,3]
[1,]  0.61404 0.05637 -0.09306
[2,]  0.05637 0.46462  0.05637
[3,] -0.09306 0.05637  0.61404
\end{lstlisting}

\bigskip
\bigskip
%----------------------------------------------------------------------------------------
%	PROBLEM 3
%----------------------------------------------------------------------------------------

\subsection*{Question 3}
Consider the matrices
$$
\mathbf{A} = 
\left( \begin{array}{cc}
  4 & 4.001 \\
  4.001 & 4.002
\end{array} \right) \ \ \ 
\text{and} \ \  \ 
\mathbf{B} = 
\left( \begin{array}{cc}
  4 & 4.001\\
  4.001 & 4.002001
\end{array} \right)
$$
Obviously, these matrices are nearly identical. Use {R} and compute the determinants and 
inverses of these matrices. (Even though the original two matrices are nearly the same, 
$\mathbf{A}^{-1} \approx -3 \mathbf{B}^{-1}$. This shows that small changes in the in 
the elements of nearly singular matrices can have big effects on some matrix operations.)

\bigskip
\subsubsection*{Answer to Question 3}
An example \texttt{R} session:
\begin{lstlisting}[basicstyle=\ttfamily\small\bfseries]
> A <- matrix(c(4 , 4.001 , 4.001, 4.002), 2)
> B <- matrix(c(4 , 4.001 , 4.001, 4.002001), 2)

> det(A)  # nearly zer0
[1] -1e-06

> det(B)  # nearly zer0
[1] 3e-06

> Ainv <- solve(A)  
> Binv <- solve(B)  

> Ainv
         [,1]     [,2]
[1,] -4002000  4001000
[2,]  4001000 -4000000

> Binv
         [,1]     [,2]
[1,]  1334000 -1333667
[2,] -1333667  1333333

> 3 * Binv # check that this is approximately Ainv
         [,1]     [,2]
[1,]  4002001 -4001000
[2,] -4001000  4000000

\end{lstlisting}
\bigskip
\bigskip
\pagebreak
%----------------------------------------------------------------------------------------
%	PROBLEM 4
%----------------------------------------------------------------------------------------

\subsection*{Question 4}

Write an \texttt{R} function to conduct projection, e.g. with name \texttt{project()}, so 
that the input is the given design matrix $\mathbf{X}$, and the output is the projection
$\mathbf{P}_{\mathbf{X}}$ for projecting a vector onto the column space of $\mathbf{X}$.

\bigskip
\subsubsection*{Answer to Question 4}
\begin{lstlisting}[basicstyle=\ttfamily\small\bfseries]
project <- function(X) {
    # Computes the projection matrix onto the column space of X
    #
    # Args:
    #   X: a design matrix
    #
    # Returns:
    #   P: the projection matrix
    
    # library MASS' ginv() computes the generalized inverse:
    suppressPackageStartupMessages(library(MASS)) 
    
    P <- X %*% ginv(t(X) %*% X) %*% t(X)
}
\end{lstlisting}
\bigskip
\bigskip

\pagebreak
%----------------------------------------------------------------------------------------
%	PROBLEM 5
%----------------------------------------------------------------------------------------

\subsection*{Question 5}

Consider the (non-full-rank) two-way ``effect model''  with interactions in the Example (d) 
in lecture. 
\begin{enumerate}[(a)]
\item Determine which of the parametric functions below are estimable:
$$
\alpha_1, \alpha_2 - \alpha_1, \mu + \alpha_1 + \beta_1 + \delta_{11}, \delta_{12},  \delta_{12} - \delta_{11} - (\delta_{22} -  \delta_{21})
$$
For those that are estimable, find $\mathbf{c}^T (\mathbf{X^T} \mathbf{X} )^- \mathbf{X}^T$, 
such that $\mathbf{c}^T (\mathbf{X^T} \mathbf{X} )^- \mathbf{X}^T \mathbf{Y}$ 
produces the estimate of $\mathbf{c}^T \beta$.

\item For the parameter vector $\beta$ written in the order used in class, consider the hypothesis $H_0 \ : \mathbf{C} \mathbf{\beta} = \mathbf{0}$ for
$$
\mathbf{C} = 
\left( \begin{array}{ccccccccc}
  0 & 1 & -1 & 0 & 0 & 0 & 0 & 0 & 0\\
  0 & 0 & 0 & 0 & 0 & 1 & -1 & -1 & 1
\end{array} \right)
$$
Is this hypothesis testable? Explain.
\end{enumerate}


\bigskip
\subsubsection*{Answer to Question 5}


\begin{enumerate}[(a)]
\item The $\matY = \matX \  \beta + \mathbf{\epsilon}$ model is of the following form here:
$$
\begin{bmatrix} 
y_{1 1 1} \\ y_{1 1 2}\\ y_{1 2 1} \\ y_{1 2 2} \\ y_{2 1 1} \\y_{2 1 2} \\ y_{2 2 1} \\ y_{2 2 2} 
\end{bmatrix}
=
\begin{bmatrix} 1 & 1 & 0 & 1 & 0 & 1 & 0 & 0 & 0 \\1 & 1 & 0 & 1 & 0 & 1 & 0 & 0 & 0\\1 & 1 & 0 & 0 & 1 & 0 & 1 & 0 & 0\\1 & 1 & 0 & 0 & 1  & 0 & 1 & 0 & 0\\1 & 0 & 1 & 1 & 0 & 0 & 0 & 1 & 0\\1 & 0 & 1 & 1 & 0 & 0 & 0 & 1 & 0\\1 & 0 & 1 & 0 & 1 & 0 & 0 & 0 & 1\\1 & 0 & 1 & 0 & 1 & 0 & 0 & 0 & 1\end{bmatrix}
\begin{bmatrix} 
\mu \\ \alpha_1\\ \alpha_2 \\ \beta_1 \\ \beta_2 \\ \delta_{11} \\ \delta_{12} \\ \delta_{21} \\ \delta_{22} 
\end{bmatrix}
+ 
\begin{bmatrix} 
\epsilon_{1 1 1} \\ \epsilon_{1 1 2}\\ \epsilon_{1 2 1} \\ \epsilon_{1 2 2} \\ \epsilon_{2 1 1} \\ \epsilon_{2 1 2} \\ \epsilon_{2 2 1} \\ \epsilon_{2 2 2} 
\end{bmatrix}
$$
In what follows we use the \texttt{project()} function from question \# 4 to determine whether any vector $\mathbf{v} \in \mathbb{R}^9$ belongs to the \emph{row} space of the design matrix $\matX$ (or equivalently, whether it belongs to the \emph{column} space of $\matX^{T}$, denoted by $ \mathbf{v} \in C(\matX^T)$; please see the attached \texttt{R} code for computational details).
\bigskip

Denoting the projection matrix onto the column space of $\matX^T$ by $\matP$, we use the following criterion to determine whether $\mathbf{v} \in C(\matX^T)$:
$$
\mathbf{v} \in C(\matX^T) \ \ \ \text{iff} \ \ \ \ \matP \mathbf{v} = \mathbf{v}.
$$

\bigskip

%------------ \alpha_1 --------
\begin{itemize}
\item $\alpha_1:$ 
Is $ (0, 1, 0, 0, 0, 0, 0, 0, 0)
\begin{bmatrix} 
\mu \\ \alpha_1\\ \alpha_2 \\ \beta_1 \\ \beta_2 \\ \delta_{11} \\ \delta_{12} \\ \delta_{21} \\ \delta_{22} 
\end{bmatrix}
$ estimable? 

\bigskip
Or, equivalently, is $\ (0, 1, 0, 0, 0, 0, 0, 0, 0) \ \ \text{in the row space of} \ \  \matX $ ?  

\bigskip
\begin{center}
\emph{Answer: \bf{No}}, because $\matP \mathbf{v} \neq \mathbf{v.}$
\end{center}
\bigskip
\bigskip

%------------ \alpha_2 - \alpha_1 --------
\item $\alpha_2 - \alpha_1:$ 
Is 
$
(0, -1, 1, 0, 0, 0, 0, 0, 0)
\begin{bmatrix} 
\mu \\ \alpha_1\\ \alpha_2 \\ \beta_1 \\ \beta_2 \\ \delta_{11} \\ \delta_{12} \\ \delta_{21} \\ \delta_{22} 
\end{bmatrix}
$ estimable? 

\bigskip
Or, equivalently, is $\ (0, -1, 1, 0, 0, 0, 0, 0, 0) \ \ \text{in the row space of} \ \  \matX $ ?  
\bigskip
\begin{center}
\emph{Answer: \bf{No}}, because $\matP \mathbf{v} \neq \mathbf{v.}$
\end{center}
\bigskip

%------------ \mu + \alpha_1 + \beta_1 + \delta_{11} --------
\item $\mu + \alpha_1 + \beta_1 + \delta_{11}:$ 
Is $
(1, 1, 0, 1, 0, 1, 0, 0, 0)
\begin{bmatrix} 
\mu \\ \alpha_1\\ \alpha_2 \\ \beta_1 \\ \beta_2 \\ \delta_{11} \\ \delta_{12} \\ \delta_{21} \\ \delta_{22} 
\end{bmatrix}
$ estimable? 

\bigskip
Or, equivalently, is $(1, 1, 0, 1, 0, 1, 0, 0, 0) \ \ \text{in the row space of} \ \  \matX $ ?  

\bigskip
\begin{center}
\emph{Answer: \bf{Yes}}, because $\matP \mathbf{v} = \mathbf{v.}$
\end{center}
The following linear combination of rows gives $(1, 1, 0, 1, 0, 1, 0, 0, 0) $:
$$
(1, 0, 0, 0, 0, 0, 0, 0) \times 
\begin{bmatrix} 1 & 1 & 0 & 1 & 0 & 1 & 0 & 0 & 0 \\1 & 1 & 0 & 1 & 0 & 1 & 0 & 0 & 0\\1 & 1 & 0 & 0 & 1 & 0 & 1 & 0 & 0\\1 & 1 & 0 & 0 & 1  & 0 & 1 & 0 & 0\\1 & 0 & 1 & 1 & 0 & 0 & 0 & 1 & 0\\1 & 0 & 1 & 1 & 0 & 0 & 0 & 1 & 0\\1 & 0 & 1 & 0 & 1 & 0 & 0 & 0 & 1\\1 & 0 & 1 & 0 & 1 & 0 & 0 & 0 & 1\end{bmatrix}
=
(1, 1, 0, 1, 0, 1, 0, 0, 0)
$$
\bigskip

%------------ \delta_{12} --------
\item $\delta_{12}:$ 
Is $(0, 0, 0, 0, 0, 0, 1, 0, 0) 
\begin{bmatrix} 
\mu \\ \alpha_1\\ \alpha_2 \\ \beta_1 \\ \beta_2 \\ \delta_{11} \\ \delta_{12} \\ \delta_{21} \\ \delta_{22} 
\end{bmatrix}
$ estimable? 

\bigskip
Or, equivalently, is $(0, 0, 0, 0, 0, 0, 1, 0, 0) \ \ \text{in the row space of} \ \  \matX $ ?  
\bigskip
\begin{center} 
\emph{Answer: \bf{No}}, because $\matP \mathbf{v} \neq \mathbf{v.}$
\end{center}
\bigskip


%------------ \delta_{12} - \delta_{11} - (\delta_{22} -  \delta_{21}) --------
\item $\delta_{12} - \delta_{11} - (\delta_{22} -  \delta_{21}):$ 
Is $(0, 0, 0, 0, 0, -1, 1, 1, -1)
\begin{bmatrix} 
\mu \\ \alpha_1\\ \alpha_2 \\ \beta_1 \\ \beta_2 \\ \delta_{11} \\ \delta_{12} \\ \delta_{21} \\ \delta_{22} 
\end{bmatrix}
$ estimable? 

\bigskip
Or, equivalently, is $(0, 0, 0, 0, 0, -1, 1, 1, -1) \ \ \text{in the row space of} \ \  \matX $ ?  
\bigskip
\begin{center} 
\emph{Answer: \bf{Yes}}, because $\matP \mathbf{v} = \mathbf{v.}$
\end{center}
The row combination is:
$$
(-1, 0, 1, 0, 0, 1, 0, -1) \times 
\begin{bmatrix} 1 & 1 & 0 & 1 & 0 & 1 & 0 & 0 & 0 \\1 & 1 & 0 & 1 & 0 & 1 & 0 & 0 & 0\\1 & 1 & 0 & 0 & 1 & 0 & 1 & 0 & 0\\1 & 1 & 0 & 0 & 1  & 0 & 1 & 0 & 0\\1 & 0 & 1 & 1 & 0 & 0 & 0 & 1 & 0\\1 & 0 & 1 & 1 & 0 & 0 & 0 & 1 & 0\\1 & 0 & 1 & 0 & 1 & 0 & 0 & 0 & 1\\1 & 0 & 1 & 0 & 1 & 0 & 0 & 0 & 1\end{bmatrix}
=
(0, 0, 0, 0, 0, -1, 1, 1, -1)
$$


\bigskip

\end{itemize}
\item For $H_0$ to be testable, each hypothesis shoud be testable (and also the $rank(\matC) = 2$). In this case, since $\alpha_1 - \alpha_2$ is not esitmable, $H_0$ is \emph{not} testable.
\end{enumerate}


\pagebreak
\section*{Appendix}

\begin{lstlisting}[basicstyle=\ttfamily\small\bfseries]
# This script helps answer questions for problem 5, assignment 2.
#
# To test whether a vector belongs to the rowspace of a design 
# matrix, project it onto that space (Equivalently, project it onto 
# the column space of X^T). If the projection is the same as the 
# vector being  projected, then it's in the space; else it's not.

X = matrix(  # The design matrix
    c(1, 1, 1, 1, 1, 1, 1, 1,
      1, 1, 1, 1, 0, 0, 0, 0,
      0, 0, 0, 0, 1, 1, 1, 1,
      1, 1, 0, 0, 1, 1, 0, 0,
      0, 0, 1, 1, 0, 0, 1, 1,
      1, 1, 0, 0, 0, 0, 0, 0,
      0, 0, 1, 1, 0, 0, 0, 0,
      0, 0, 0, 0, 1, 1, 0, 0,
      0, 0, 0, 0, 0, 0, 1, 1), 
    nrow=8, 
    ncol=9) 

# load the project() function from problem 4:
source("project.R")
P <- project(t(X))

#------------------------------------------------
# question: is alpha1 estimable?
alpha1 <- matrix(c(0, 1, 0, 0, 0, 0, 0, 0, 0))
print(P %*% alpha1)        #not estimable

#------------------------------------------------
# question: is alpha2 - alpha1 estimable:
alpha2_1 <- matrix(c(0, -1, 1, 0, 0, 0, 0, 0, 0))
print(P %*% alpha2_1)      # not estimable 

#------------------------------------------------
mu_and_others <- matrix(c(1, 1, 0, 1, 0, 1, 0, 0, 0))
print(P %*% mu_others)     # estimable

#------------------------------------------------
delta1_2 <- matrix(c(0, 0, 0, 0, 0, 0, 1, 0, 0))
P %*% delta1_2

#------------------------------------------------
deltas  <- matrix(c(0, 0, 0, 0, 0, -1, 1, 1, -1))
P %*% deltas               # estimable 
\end{lstlisting}
















\end{document}